\documentclass{beamer}
 
\usepackage[utf8]{inputenc}
\usepackage{graphicx}
\usepackage{xcolor}
\usepackage{}

\usetheme{Boadilla}
\usecolortheme{beaver}

%Information to be included in the title page:
\title{Facebook Project Simulations}
\author{James Li}
\date{1 September 2020}

\begin{document}
\begin{frame}{Data Generating Process}

Objective for DGPs:
\begin{itemize}
    \item Create multiple clusters, where distributions of covariates are different across clusters
    \item Each cluster has a different arm that produces highest reward
    \item Generate "lumpy" reward functions that cannot be straightforwardly recovered by a linear model
    \item Allow levers to move:
    \begin{itemize}
        \item number of covariates used to define clusters
        \item relative size of clusters
        \item \textit{Heterogeneity ratio} (value of best contextual/best fixed policy)
    \end{itemize} 
\end{itemize}
\end{frame}


\begin{frame}{Data Generating Process}
Requirements for DGPs:
\begin{itemize}
    \item The difference between the best contextual policy \& the control is fixed across DGPs
    \begin{itemize}
        \item[$\rightarrow$] Differences in power curves between DGPs are based on ability of agent to learn the DGP, not differences in effect sizes
    \end{itemize}
    \item The best fixed arm is always the same arm across DGPs
\end{itemize}

\end{frame}

\begin{frame}{Data Generating Process}

\textbf{Generate Baseline Dataset ($N = 10000, p = 15$)}
\begin{itemize}
    \item Parameter \begin{itemize}
        \item Number of useful covariates $p' \in \{3, 5, 10\}$
        \item Largest cluster size ratio $c \in \{0.4, 0.6, 0.8\}$
        \item Heterogeneity ratio $h \in \{1.05, 1.5, 1.95\}$
    \end{itemize}
    \item Generate large covariate matrix $X$ using correlated multivariate normal distribution, covariance matrix generated from $\frac{Beta(2,2) - 0.5}{2}$
    \item Use iterative KNN to group covariates observed into $k = 3$ clusters with cluster size [$N*c, \frac{N * (1-c)}{2}, \frac{N * (1-c)}{2}$], where $c$ = largest cluster size ratio
    
\end{itemize}

\end{frame}

\begin{frame}{Data Generating Process}
\textbf{Reward Generation}
\begin{itemize}
    \item Generate reward for best arm for each cluster $c_i,  i = 1, .., 3$, $$R_{w_{best, i}, c_i} = 0.6 - X_{1, c_i}$$
    \item Reward for 2nd best arm for each cluster $c_i,  i = 1, .., 3$, $$R_{w_{best_2, i}, c_i} = R_{w_{best, i}, c_i} + \epsilon $$ where $\epsilon \sim \mathcal{N}(\mu, 0.01), \mu <= 0$. 
    \item Reward for the rest of the arm
    $$R_{w, c_i} = -X_{1, c_i} \mbox{ for }  w \not\in \{w_{best, i}, w_{best_2, i}\}, i = 1, .., 3,$$
    \item We vary $\mu$ to search for the level of desired heterogeneity ratio $h$.
\end{itemize}
\end{frame}

\begin{frame}{Data Generating Process}
\textbf{Note}
\begin{itemize}
    \item 0.6 was chosen to simulate an average treatment effect between best-fixed policy and control policy; assume control policy is not the best arm for any cluster. (See PAP for justification of magnitude.)
    \item In the simulations, arm 0 is chosen as the best arm for the largest cluster. It is also chosen as the 2nd best arm for the other two clusters to ensure it is the best-fixed policy.
\end{itemize}
\end{frame}

\end{document}